\documentclass{article}
\usepackage[a4paper, total={6in, 9in}]{geometry}
\usepackage[utf8]{inputenc}
\usepackage{authblk}
\renewcommand\Authand{, y }
\renewcommand\Authands{, y }

% Definimos la fuente Lato
\usepackage[default]{lato}

% Títulos automáticos en español
\usepackage[spanish]{babel}

% Soporte para buenas urls e hipervínculos entre secciones
\usepackage{hyperref}
\usepackage{url}

% Imágenes y figuras
\usepackage{graphicx}

% Código fuente con números de línea
\usepackage{listings}
\usepackage{xcolor}

% Email links
\newcommand*{\email}[1]{\href{mailto:#1}{\nolinkurl{#1}} } 

%Definimos los colores para el código
\definecolor{aurometalsaurus}{rgb}{0.43, 0.5, 0.5}
\definecolor{codegray}{rgb}{0.5,0.5,0.5}
\definecolor{mediumpersianblue}{rgb}{0.0, 0.4, 0.65}
\definecolor{backcolour}{rgb}{0.95,0.95,0.92}
\definecolor{green(ncs)}{rgb}{0.0, 0.62, 0.42}

%Definimos el estilo para el código
\lstdefinestyle{mystyle}{
  backgroundcolor=\color{backcolour},   commentstyle=\color{aurometalsaurus},
  keywordstyle=\color{mediumpersianblue},
  numberstyle=\tiny\color{codegray},
  stringstyle=\color{green(ncs)},
  basicstyle=\ttfamily\footnotesize,
  breakatwhitespace=false,         
  breaklines=true,                 
  captionpos=b,                    
  keepspaces=true,                 
  numbers=left,                    
  numbersep=5pt,                  
  showspaces=false,                
  showstringspaces=false,
  showtabs=false,                  
  tabsize=2
}
\lstset{style=mystyle}

%----------------------------------------------------------
%              TÍTULO DEL DOCUMENTO Y AUTORES
%----------------------------------------------------------
\title{Node-RED}
\author{Lucía Atienza Olmo

Carlos González Parrado}

%----------------------------------------------------------

\date{}
\affil{Sistemas Distribuidos\\ Grado en Ingeniería Informática\\ Universidad de Cádiz}

%----------------------------------------------------------
%               INICIO DEL DOCUMENTO
%----------------------------------------------------------
\begin{document}
\maketitle


% Sustituir el texto de ejemplo con el del propio trabajo
\section{Flujo 1: Procesador de cadenas}
\bigskip

La API Rest 'Procesador de Cadenas' contiene los endpoints básicos mas las mejoras propuestas en la práctica, como se explicará a continuación.
\bigskip

Nota: el proceso de transmisión de mensajes es levemente ampliado por la mejora 'Creación de un log', ya que intervendrán dos nodos extras en todos los endpoints: el nodo función 'Normalizar' y nodo file 'Log\_PdC', aunque no modifican su comportamiento, y por ello lo describimos en su apartado (1.2.2).

\subsection{Endpoits básicos de la API Rest}
\begin{itemize}
\item \textbf{/reverse}: como se indica, usa el método GET. Recibe una cadena, y devolverá su inversa en el mismo formato (json, con el mismo nombre, 'cadena'). El proceso para hacerlo es simple: recibimos la petición a través del nodo 'http in', el cual envia el mensaje al nodo función 'Girar'. Girar invierte la cadena con un único bucle, recorriendo la cadena original de forma inversa y guardando los elementos en la cadena a devolver. Finalmente, pasa su resultado al nodo http response.

\item \textbf{/change\_case}:  usa el método GET. Recibe dos cadenas, 'cadena' y 'modo', y devolverá solo el campo 'cadena' tras haberle aplicado la modificación descrita. Internamente consta de 3 nodos; http in (recibe la petición), nodo función ('Mayus/Minus', transformando la cadena) y http response (envía el resultado). El nodo función simplemente comprueba el campo 'modo': si es "mayus" aplicará el método '.toUpperCase' a la cadena recibida, o el método '.toLowerCase()' en caso contrario. Una vez aplique el cambio, devolverá el resultado al nodo http response.

\item \textbf{/random}: usa el método GET. Recibe un entero, 'longitud' y devolverá una cadena compuesta por esa cantidad de caracteres aleatorios. De nuevo está formado por 3 nodos; http in, nodo función ('Random') y http response. Random simplemente lee ese parámetro recibido (longitud) y genera esa cantidad de números aleatorios (del 0 al 61, que se corresponden con los caracteres del abecedario en minuscula y mayuscula y los números del 0 al 9, almacenados en 'cars'), y guarda sus correspondientes caracteres en la cadena resultado, siendo la que devolverá. Ésta cadena se pasa al nodo http response, finalizando el flujo del endpoint.
\end{itemize}

\newpage
\subsection{Endpoits extras y mejoras de la API Rest)}
\begin{itemize}
\item \textbf{Soporte de errores}: Se usa el nodo catch 'Tratamiento de Errores', el cual recibirá mensajes de error provenientes de todos los demás nodos del flujo. Si lo recibe, pasará dicho mensaje a la función 'Mensaje al usuario' (proporciona un poco mas de información en el payload del mensaje), y éste se lo comunica al nodo http response (notifica al usuario) y también a un nodo debug (para poder ver la información desde el editor de node-red).

\item \textbf{Creación de un log}: Se usa el nodo función 'Normalizar', que simplemente prepara el contenido del mensaje, añadiendo al resultado devuelto por las funciones sus parámetros origen ('cadena', 'longitud'...dependiendo del endpoint usado) y el resto de campos requeridos para el log. Finalmente, esta información se envia al nodo file ''Log\_PdC', el cual creará dicho log en el directorio raíz del usuario.

\item \textbf{/diff}: Dados dos parametros, 'cadenaA' y 'cadenaB', se devolverá un tercer parámetro, 'cadena', el cuál será la cadena A quitando todas las coincidencias de B. Si nuestras cadenas origen son 'Hello world Hola mundo' (A) y ' ' (B), la cadena resultante será 'HelloworldHolamundo'. El flujo es similar a los endpoints anteriores, y la función implementada en 'Diferencia' consiste en un bucle que recorre la cadena A, copiando sus caracteres en la cadena resultado (C); si se da la coincidencia del primer caracter, se comprueba que el substring comprendido entre esa posición i e i+Tamaño de cadena B coincidan. Si es así, saltamos toda esa distancia de la cadena y seguimos copiando los siguientes caracteres.

\item \textbf{/dni}: Con un flujo similar a los anteriores, la función 'DNI' recoge el parámetro 'cadena' (que representa un DNI) y comprueba si se corresponde a un DNI válido. Si es así, devolverá en el parámetro 'resultado' la frase "SI es un DNI valido.", o "NO es un DNI valido." en caso contrario. La comprobación consiste en dividir la cadena en dos partes (los numeros y luego la letra), convertir lo primero  a formato numerico, y dividir entre 23. Esto nos dará el índice para buscar en las letras posibles del DNI; si la letra indicada es la que figura en nuestra cadena, es un DNI válido.

\item \textbf{/cesar}: El cifrado César es una manera sencilla de codificación; se desplazan las letras del mensaje original 3 posiciones (en la versión clásica de esta codificación) a la derecha del abecedario (consideramos el abecedario sin vocales con tilde, ni 'ñ' ni simbolos especiales ni numéricos; en caso de que los hubiese se les atribuye el símbolo '?'). Nuestra función 'Cifrado' recibe una 'cadena' que contiene el mensaje a cifrar, y devuelve como parámetro otra 'cadena' con el mensaje cifrado. La función consta de una cadena de caracteres con el abecedario ordenado, en minúsculas y mayúsculas, y al buscar una letra, tomaremos su coincidencia + 3 posiciones. En caso de ser un simbolo especial, al no estar en la cadena, se sustituye por un '?'. 
\end{itemize}

%\subsection{Plantilla para Curls usadas para probar las mejoras:} % Se ve raro...
%
%\begin{itemize}
%\item \textbf{DNI}: curl -X GET -d '\{ \" cadena": "49566252X"\}' -H " Content-type: application/json"  http:/localhost:1880/dni
%
%\item \textbf{Diferencia}: curl -X GET -d '\{" cadenaA": "Hallo world", "cadenaB": "o"\}' -H " Content-type: application/json"  http:/localhost:1880/diff
%
%\item \textbf{César}: curl -X GET -d '\{" cadena": " ABCdef"\}' -H " Content-type: application/json"  http:/localhost:1880/cesar
%
%\item \textbf{Error en /reverse} :curl -X GET -d '\{" cadena": 1234567890\}' -H " Content-type: application/json"  http:/localhost:1880/reverse
%
%\item \textbf{Error en /random}: curl -X GET -d '\{" longitud": "AA" \}' -H " Content-type: application/json" http:/localhost:1880/random
%\end{itemize}



\newpage
\section{Flujo 2: MQTT}



\newpage
\section{Flujo 3: Contributed Nodes}

%----------------------------------------------------------
%               FIN DEL DOCUMENTO
%----------------------------------------------------------

% Sección de referencias
\section*{Referencias}
\begin{enumerate}

 \item Seminario 5 de la asignatura, Node-RED - Partes 1 y 2
 \item Conversión de cadena, toUpperCase(); 

 https://developer.mozilla.org/en-US/docs/Web/JavaScript/Reference/Global\_Objects/String/toUpperCase
 \item Objeto Math, https://developer.mozilla.org/en-US/docs/Web/JavaScript/Reference/Global\_Objects/Math
 \item Cifrado César, https://es.wikipedia.org/wiki/Cifrado\_C\%C3\%A9sar
 \item Tratamiento de errores, https://nodered.org/docs/user-guide/handling-errors
\end{enumerate}


\end{document}
